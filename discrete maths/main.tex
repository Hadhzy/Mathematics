\section{Mathematical thinking}
\section{Variables}
A variable is sometiems thought of as a mathematical "John Doe" because you can use it as a placeholder when you want to talk about 
something but either you imagine that it has one or more values but you don't know what they are, or you want whatever you say
about it to be equally true for all elements in a given set. This can be easily recognised/understood in programming languages where variables play a vital role and 
fundemental part of the language. It has slighly different meanings in mathematics and computer science. When one uses variable in mathematical terminology it can be defined
as a symbol that represents a quantitiy in a mathematical expresssion, as used in many sciences. In computer science it is rather something storying a value and this associated value may be changed(mutable).


\section{Mathematical statements}
Three of the most important kinds of sentences in mathematics are universal statements, conditional statements, and existential statements.

\textbf{Universal statement}: A universal statement says that a certain property is true for all elements in a set. 
\textbf{Conditional statement}: A conditional statemetn says that if one thing is true then some other thing also has to be true.
\textbf{Existential statement}:   An existential statement says that there is at least one thing for which the propert is true.


In other words,
\textbf{Universal statement}
\textbf{Conditional statement}
\textbf{Existential statement}: