\section{Supervised and Unsupervised Learning}

\subsection{Supervised Learning1}
Supervised learning is a type of machine learning algorithm that uses a known dataset (called the training dataset) to make predictions. The training dataset includes input data and output data. The key of supervised learning is that you give your learning algorithm examples to learn from that include the "right answers", in other words the correct label \textbf{y} for the input \textbf{x}.
It learns by seeing correct pairs of inputs and outputs.  The goal is that eventually the algorithm will only be provided with the input-value(x) and it has to predict the output-value/label(y). 
Examples: \\ \\ \\

\begin{align*}
    \scalebox{1.3}{
\begin{tabular}{|c|c|c|}
    \hline
    Input(x) & Output(y) & Application \\
    \hline
    email & spam?(0/1) & spam filtering \\
    \hline
    audio & text transcript & speech recognition \\
    \hline
    English & Spanish & machine translation \\
    \hline
    ad,user,info & click?(0/1) & online advertising \\
    \hline
    image, radar info & position of other cars & self-driving car \\
    \hline
    image of a phone & defect?(0/1) & visual inspection \\
    \hline
\end{tabular}}
\end{align*}

In all of these applications you first train your model on data where you know the correct output. Then you use that model to make predictions on new data. \\

\subsubsection*{Regression: Housing Price Prediction}
Regression is a type of supervised learning algorithm that tries to predict a continuous output variable. In other words, it is used to predict a number. For example, predicting the price of a house in dollars is a regression problem whereas predicting whether a tumor is malignant or benign is a classification problem. \\
In other words, Regression is a type of algorithm that tries to predict a continuous output variable. 

\subsubsection*{Classification: Breast Cancer Prediction}
Say that you are trying to create a model that predicts whether or not a tumor is malignant or benign. In classification problems, the output variable is a category (such as "malignant" or "benign"). What we are trying to do is map input variables to discrete categories, we are classifying the input values into categories and trying to predict from the two possible output. Here we have a number of possible categories or outputs that our algorithm can predict. Whereas with regression we are trying to predict a continuous number. 
To summarise, Classification is a type of algorithm that tries to predict a discrete category, that doesn't necessarily have to be a number. Also in classification we are predicting a small set of possible outputs, so we are given our "options" in regards of what can our output be.
By extension obviously, we can have two or more inputs, in this case we can think of inputs as a "context" dataset that is provided to us, such as the tumor size, or the patient's age, sex, etc... The more context we provide the more accurate our prediction will be.
What we often do is represent our data-set in a function for example and our algorithm would try to identify some form of pattern in the data-set.

\subsection{Summary of Supervised Learning}
In summary supervised learning is when we are given a data-set and we are told what our correct output should be. We are given the "right answers" and we are trying to learn from that data-set to be able to predict the correct output for new data without the need of the input.
We have looked at two categories of supervised learning, regression and classification. In regression we are trying to predict a continuous number, and in classification we are trying to predict a discrete category.


\subsection{Unsupervised Learning}

Previously, in our classification problem what we have looked at is a supervised learning problem. We have a data-set and we are given the correct output(y) for each example. In unsupervised learning we are given a data-set but we are \textbf{not} given any labels or output value(y). Say you were given with the tumor size and the patient's age and any other context, however you don't know whether the tumor is malignant or benign. You are not given any labels or output value. 
Our job here is not to predict whether the tumor is malignant or benign, but rather to find some structure in the data-set. In other words find something interesting in the unlabeled data-set. 
The reason why it is called unsupervised is because we are not supervising the algorithm, we are not telling it what the correct answer is, we are just giving it the data and asking it to find some structure in the data.
So it might identify that the data can be divided into number of sections, or it might find some form of patterns between the patient's age and their tumor size, or any other contextual prediction.
Essentially it tries to place the data into different clusters, clustering is a type of unsupervised learning, where we are trying to find some structure in the data. For example clustering is used in google news. 
What google news does is every day it goes through all the news articles that are published and it tries to group them into different clusters, so that it can show you the different clusters of news articles and you can choose which cluster you are interested in.

Another type of unsupervised learning is anomaly detection where we are trying to find some abnormal behavior in the data. For example, if we have a data-set of credit card transactions and we are trying to find some unusual behavior in the data-set, such as unusual credit card transactions.
Another example is in manufacturing, where we are trying to find some unusual behavior in the manufacturing process, such as unusual defects in the product.
We also have dimensionality reduction where we are trying to take a data-set with a large number of features and reduce the number of features to a smaller number. For example, if we have a data-set with a large number of features, we might want to reduce the number of features to a smaller number so that we can visualize the data more easily.

\subsection{Linear Regression: House sizes and prices}
In this example let's imagine that we are given with a data-set about house prices and their sizes. We are given the size of the house in square feet and the price of the house in dollars. We are given a number of examples of houses and their prices. We are trying to predict the price of a new house given its size.
This is a perfect example of a linear regression model where we would first train our model on the data-set and showing the "right answers" and then we can use that model to predict the price of a new house given its size.

A data-set that is used to train a model is called a training set. The sizes of the house can be represented with x as for example if we  have a house that is 1000 square feet, we can represent that with x = 1000. The price of the house can be represented with y, so if the price of the house is 200,000 dollars, we can represent that with y = 200,000. 

We call x as the "input" variable or feature, and refer to y as the "output" variable or target. 

We can represent as a single training example by $$(x,y)$$ where x is the input and y is the output.
Our $i$th training example where I would represent a specific row from the training set would look like this: $(x^{(i)},y^{(i)})$.

Note that this doesn't mean that we are taking x and y to the power of i, but rather that we are taking the i'th training example from the training set.

So for example one example would look like this $$ (x^{(1)}, y^{(1)})=(2104,400) $$

Also as mentioned $$x^{(2)}\neq x^2$$

\subsection{Process of a Supervised training}
Supervised learning algorithm will input a dataset and then what exactly does it do and what does it output? 
Recall that a training set in supervised learning includes both the input features, such as the size of the house and also the output targets, such as the price of the house. The output targets are the right answers to the model we'll learn from.
To train the model, you feed the training set, both the input features and the output targets to your learning algorithm. 