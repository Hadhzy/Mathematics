\section{Supervised and Unsupervised Learning}

\subsection{Supervised Learning}
Supervised learning is a type of machine learning algorithm that uses a known dataset (called the training dataset) to make predictions. The training dataset includes input data and output data. The key of supervised learning is that you give your learning algorithm examples to learn from that include the "right answers", in other words the correct label \textbf{y} for the input \textbf{x}.
It learns by seeing correct pairs of inputs and outputs.  The goal is that eventually the algorithm will only be provided with the input-value(x) and it has to predict the output-value/label(y). 
Examples: \\ \\ \\

\begin{align*}
    \scalebox{1.3}{
\begin{tabular}{|c|c|c|}
    \hline
    Input(x) & Output(y) & Application \\
    \hline
    email & spam?(0/1) & spam filtering \\
    \hline
    audio & text transcript & speech recognition \\
    \hline
    English & Spanish & machine translation \\
    \hline
    ad,user,info & click?(0/1) & online advertising \\
    \hline
    image, radar info & position of other cars & self-driving car \\
    \hline
    image of a phone & defect?(0/1) & visual inspection \\
    \hline
\end{tabular}}
\end{align*}

In all of these applications you first train your model on data where you know the correct output. Then you use that model to make predictions on new data. \\