\section{Linear Equations}

A linear equation is an equation in which the highest power of the variable is always 1. It is also known as a one-degree equation. 
Additionally A linear equation is an equation that can be written as the sum of numbers(coefficients) times variables, added up to equal to a number.

For example. $ 2x + 3y + 7.3z = 5 $ is a linear equation.

\begin{itemize}
    \item $4x^2 + 5x + 6 = 0$ is not a linear equation because the highest power of the variables is 2.
    \item $ \frac{1}{3}uv + \frac{2}{3}u = \frac{3}{5}v $ is not a linear equation because it includes a product of variables.
    \item $ 5a + 2 = 6b - \sqrt[]{2c}  $ is a linear equation because if we rearrange it, we will get: 
    $$ 5a - 6b + \sqrt[]{2c} = -2 $$
\end{itemize}

a system of linear equations (or linear system) is a collection(set) of one or more linear equations involving the same variables

For example, 
\begin{align*}
    3x + 2y - z &= 1 \\
    2x - 2y + 4z &= -2 \\
    -x + \frac{1}{2}y -z &= 0
\end{align*}
is a system of three equations in the three variables $x, y, z$. A solution to a linear system is an assignment of values to the variables such that all the equations are simultaneously satisfied. \\
A solution to the system above is given by the ordered tuple.
$$ {\displaystyle (x,y,z)=(1,-2,-2)} $$  
\subsection{Solving systems of Linear Equations with the method of substitution}
It is important to establish that there are limitless ways to solve systems of linear equations. 
Example. Solve the system of linear equations(we are going to use the method of substitution): 
\begin{align*}
    -2a + 3b + 4c = 1 \\
    a + b + 5c = 2 \\
    b = 2a + c
\end{align*}
Firstly we already have a value for $b$ in the 3rd equation so let's just substitute it into the first two equations.
\begin{align*}
    &1. \quad -2a + 3(2a + c) + 4c = 1 \\
    &2. \quad a + (2a + c) + 5c = 2 \\ 
    &3. \quad  b = 2a + c   
\end{align*}

Let's simplify the first two equations.
After expanding and simplifying the first two terms we will get:
\begin{align*}
    &1.\quad 4a + 7c = 1 \\
    &2.\quad 3a + 6c = 2 \\
    &3.\quad b = 2a + c      
\end{align*}
Notice that now we have two equations with two variables, so we can solve for $a$ and $c$.
\\ 

\begin{itemize}
    \item In the first equation we can subtract $7c$ and then divide by $4$ to get $a$. Which will give us: 
    \begin{align*}
        1. \quad  a = \frac{1}{4} - \frac{7}{4}c \\
        2. \quad 3a + 6c = 2 \\
        3. \quad b = 2a + c
    \end{align*}
    \item Now that we have the value for $a$ we can substitute that into the second equation. Which will give us:
    \begin{align*}
        &1. \quad a = \frac{1}{4} - \frac{7}{4}c \\
        &2. \quad 3(\frac{1}{4} - \frac{7}{4}c) + 6c = 2 \\
        &3. \quad b = 2a + c
    \end{align*}
    \item Now let's simplify the second equation. After expanding and simplifying with the like terms we will get:
    \begin{align*}
        &1. \quad a = \frac{1}{4} - \frac{7}{4}c \\
        &2. \quad \frac{3}{4} - \frac{21}{4}c + 6c = 2 \xrightarrow{\text{add like terms}} \frac{3}{4} + \frac{3}{4}c = 2\\
        &3. \quad b = 2a + c
    \end{align*}
    \item Now that we have this, we can simply subtract $\frac{3}{4}$ then simplify it down to: 
    $$ c = \frac{5}{3} $$ 
    \item Since we have the value for $c$ now let's figure out the value for $a$.
    $$ a = \frac{1}{4} - \frac{7}{4}(\frac{5}{3}) = \frac{1}{4} - \frac{35}{12} = - \frac{32}{12} = -\frac{8}{3}$$
    Therefore, $ a = - \frac{8}{3} $
    \item Now that we have the value for $a$ and $c$ we can substitute that into the third equation to get the \textbf{actual} value for $b$.
    $$ b = 2(-\frac{8}{3}) + \frac{5}{3} = -\frac{16}{3} + \frac{5}{3} = -\frac{11}{3}$$
\end{itemize}

Now finally we have solved the system of linear equations and we have the values for $a, b, c$. If we'd plug them in into the original equations then it would satisfy all of them. \\ 
We can list the values in an ordered tuple like this:
$$ {\displaystyle (a,b,c)=(-\frac{8}{3},-\frac{11}{3},\frac{5}{3})} $$
Now let's see another way of solving systems of linear equations.
\subsection{Solving systems of Linear Equations with the method of elimination}
Example. Solve the system of linear equations(we are going to use the method of elimination): 
\begin{align*}
    &1. \quad -2a + 3b + 4c = 1 \\
    &2. \quad a + b + 5c = 2 \\ 
    &3. \quad  b = 2a + c   
\end{align*}
Let's rearrange this system by writing each equation in the standard form, where all the variables are on the left side and the constants are on the right side.
The first two equations are already in the standard form, but the third one is not. So let's rearrange it.

\begin{align*}
    &1. \quad -2a + 3b + 4c = 1 \\
    &2. \quad a + b + 5c = 2 \\ 
    &3. \quad  -2a + b - c = 0  
\end{align*}