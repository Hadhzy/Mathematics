\section{Formal Grammars}
The above recipe form, based on replacement according to rules, is strong enough
to serve as a basis for formal grammars. Similar forms, often called “rewriting systems”, have a long history among mathematicians, and were already in use several
centuries B.C. in India (see, for example, Bhate and Kak [411]). The specific form
of Figure 2.3 was first studied extensively by Chomsky [385]. His analysis has been
the foundation for almost all research and progress in formal languages, parsers and
a considerable part of compiler construction and linguistics.

\subsection{Formalism of Formal Grammars}
A formal grammar is a set of rules for generating a language. Work in this field has been done with a specific notation. 
We shall start with some mathematic notation, which is used for correctness and proofs and not necessarily to understand the principles.

\begin{definition}
    A \textbf{generative grammar} is a 4-tuple $(V_N, V_T, R, S)$ such that \\
    (1) $V_N$ and $V_T$ is a finite set of symbols \\ 
    (2) $V_N \cap V_T = \emptyset$ \\
    (3) $R$ is a finite set of pairs $(P,Q)$ such that \\
        (3a) $P \in (V_N \cup V_T)^+$ \\
        (3b) $Q \in (V_N \cup V_T)^*$ \\
    (4) $S \in V_N$ is the start symbol
\end{definition}

Explanation of the above definition: 
$V_N$ is a finite set of non-terminal symbols. $V_T$ is a finite set of terminal symbols. The two sets are disjoint, meaning that they their intersection is an empty set.
This makes sense as terminals and non-terminals should have no elements in common. \\
$R$ is a finite set of pairs $(P,Q)$, and is the set of all rules. $P$ and $Q$ are strings of terminals and non-terminals. Each $P$ by the definiton  must consist of sequences of one or more non-terminals
and terminals and each $Q$ must consist of sequences of zero or more non-terminals and terminals.  \\
Therefore our $R$ grammar set looks like: 
$$R = {(Name, tom), (Name, dick), (Name, harry), \\
(Sentence, Name), (Sentence, List End), (List, Name), \\
(List, List , Name), (, Name End, and Name)}$$

For our start symbol $S$ we state that it must be an element of $V_N$ so it must be a non-terminal: 
$$S=\textnormal{Sentence}$$
This concludes our field trip into formal linguistics. In short, the mathematics of
formal languages is a language, a language that has to be learned; it allows very concise expression of what and how but gives very little information on why. Consider
this book a translation and an exegesis.

\subsection{Generating Sentences from a Formal Grammar}
For the above grammar we can introduce a more compact notation. The right-hand sides seperated by vertical bars "|" are called alternatives. 
Therefore our grammar becomes: 

\begin{enumerate}
    \item $\text{Name} \rightarrow \text{tom} \mid \text{harry} \mid \text{bob} $
    \item $\text{Sentence} \rightarrow \text{Name} \mid \text{List End} $
    \item $\text{List} \rightarrow \text{Name} \mid \text{Name, List} $
    \item $\text{', Name End'} \rightarrow \text{and Name} $
\end{enumerate}


\subsection{Exercises}

\begin{enumerate}
    \item By the definiton of a generative grammar, why must $V_N$ and $V_T$ be disjoint?
    \item What will happen if $V_N$ and $V_T$ are not disjoint and they have intersected element(s)?
    \item What is the difference between $V_N$ and $V_T$?
    \item What is the difference between $P$ and $Q$?
    \item What does $R = {(P,Q)}$ mean?
    \item Define direct and indirect recursion. Provide examples as well.
    \item Explain the significance of the end marker. Why is it important?
    \item What is the notation to distinguish between the nonterminal and the terminal symbols?
\end{enumerate}