\section{Solving Rational Equations}
A Rational equation is an equation that contains one or more rational expressions. A rational expression is a fraction whose numerator and denominator are polynomials. For example, $\frac{x^2+1}{x+1}$ is a rational expression as discussed in the previous section. In this section we will go over how to solve rational equations.

\subsection{Solving Rational Equations}

\textbf{Example.} Solve: $\frac{x}{x+3} = 1+\frac{1}{x}$

The steps to solve a rational equation are as follows:
\begin{enumerate}
    \item Find the least common denominator (LCD) of all the fractions in the equation. In our example we have two fractions, $\frac{x}{x+3}$ and $\frac{1}{x}$, we can think of 1 as $\frac{1}{1}$. \\
    By multiplying them together, therefore the LCD is $(x+3)x$.
    \item Next step is to "clean" the denominator  by multiplying both sides of the equation by the LCD. So:
    $$ (x+3)x \cdot \frac{x}{x+3} = (x+3)x \cdot \left(1+\frac{1}{x}\right)$$
    The right side of the equation can be simplified as:
    $$ (x+3)x \cdot \left(1+\frac{1}{x}\right) = (x+3)x + (x+3)$$
    Therefore we can rewrite the equation as:
    $$ \cancel{(x+3)}x \cdot \frac{x}{\cancel{x+3}} =  (x+3)x + (x+3)$$
    $$ x^2 =  (x+3)x + (x+3)$$
    \item Third step is to simplify and solve. 
    $$ x^2 =  x^2+3x+x+3$$
    $$ \cancel{x^2} =  \cancel{x^2}+3x+x+3$$
    $$0 =  4x+3$$
    $$4x =  -3$$
    $$x =  -\frac{3}{4}$$
    \item It is a good idea to check your answer by plugging it back into the original equation. In our example after plugging it in, we have:
    $$ \frac{-\frac{3}{4}}{-\frac{3}{4}+3} = 1+\frac{1}{-\frac{3}{4}}$$
    $$ \frac{-\frac{3}{4}}{\frac{9}{4}} = 1-\frac{4}{3}$$
    $$ -\frac{3}{9} = -\frac{1}{3} $$
    Therefore we can conclude that our answer is correct. Our final answer is $x = -\frac{3}{4}$.
\end{enumerate}


\textbf{Example.} Solve: 
\begin{equation*}
    \frac{4c}{c-5} - \frac{1}{c+1} = \frac{3c^2+3}{c^2-4c-5}
\end{equation*}

The steps are the same: 
\begin{enumerate}
    \item Find the LCD of all the fractions in the equation. In our example we have three fractions, $\displaystyle \frac{4c}{c-5}$, $\displaystyle  \frac{1}{c+1}$, and $\displaystyle \frac{3c^2+3}{c^2-4c-5}$. when simplifying the 3rd one to $(c-5)(c+1)$ by factoring, the LCD is $(c-5)(c+1)$.
    \item Next step is to "clean" the denominator  by multiplying both sides of the equation by the LCD. So:
    $$ (c-5)(c+1) \cdot \frac{4c}{c-5} - (c-5)(c+1) \cdot \frac{1}{c+1} = (c-5)(c+1) \cdot \frac{(3c^2+3)}{(c-5)(c+1)}$$
    Cancelling them we have:
    $$ \cancel{(c-5)}(c+1) \cdot \frac{4c}{\cancel{c-5}} - (c-5)\cancel{(c+1)} \cdot \frac{1}{\cancel{c+1}} = \cancel{(c-5)}\cancel{(c+1)} \cdot \frac{(3c^2+3)}{\cancel{(c-5)}\cancel{(c+1)}}$$
    \item Third step is to simplify and solve So: 
    $$ 4c(c+1) - (c-5) = 3c^2+3$$
    Let's expand the left side of the equation:
    $$ 4c^2+4c - c + 5 = 3c^2+3$$
    After subtracting $3c^2$ and simplifying we have:
    $$ c^2+3c+5 = 3$$
    $$ c^2+3c+2 = 0$$
    Here we are with a quadratic equation, that we can solve by using what we have learned. 
    So we are looking for two numbers that multiply to 2 and add to 3. The numbers are 1 and 2. So we can factor the equation as:
    $$ (c+1)(c+2) = 0$$
    Since either $c+1$ or $c+2$ is equal to 0, we have two solutions: $c = -1$ and $c = -2$.
    \item When we are check our answer by plugging it back into the original equation, we notice that $c=-1$ is not a solution. Because it makes the denominator of the second fraction equal to 0. This is what we call an \textbf{extraneous solution}(In mathematics, an extraneous solution is a solution, such as that to an equation, that emerges from the process of solving the problem but is not a valid solution to the problem).
    Therefore our final answer is $c = -2$.
\end{enumerate}
The steps to solve a rational equation are as follows:
\begin{enumerate}
    \item Find the least common denominator (LCD) of all the fractions in the equation. 
    \item Then "clear" the equation of fractions by multiplying both sides of the equation by the LCD.
    \item  Simplify and solve.
    \item Check your answer by plugging it back into the original equation and eliminate any extraneous solutions(make the denominator 0).
\end{enumerate}