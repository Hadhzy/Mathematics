



\section{Summations}
In mathematics, summation is the addition of a sequence of any kind of numbers, called addends or summands; the result is their sum or total. 
Let's say that we have a sequence of numbers and we want to add them up.

\subsubsection{Summation notation - Sigma notation}
$$  \sum_{i=1}^{n} a_i = a_1 + a_2 + a_3 + \ldots + a_n $$
\begin{itemize}
  \item $ i $ = index of summation
  \item $ n $ = upper limit of summation
\end{itemize}
To understand this let's consider this: 
$$ \sum_{i=1}^{4} i = 1 + 2 + 3 + 4$$
\begin{itemize}
  \item $ i = 1 $, $ i $ is the index of summation, meaning that the number where the summation starts from(in this case $ i = 1 $ so $1$).
  \item $ 4 $ is the upper limit of summation, meaning that the number where the summation ends at(in this case $ 4 $). It is important to note that the upper limit only indicates the limit for the index $i$ and doesn't indicate the number of terms that will be included in the summation.
  \item $ i $ is the formula/rule of summation. Meaning that an expression that will be used in our sequence of numbers. Quite literally the expression that will be used in the summation. This can be as complex as we would like it to be.
\end{itemize}

Let's break down the steps of the summation notation:
\begin{itemize}
  \item Firstly we start at whatever the index is, in this case it is $1$ so we start from there. 
  \item Set $i$ equal to one and then write the $1$ down. 
  \item Then we increment the index $i$, and again writing $i$ down and then summing each of these terms as we go.
  \item We continue this process until we reach the upper limit of summation, in this case it is $4$.
  \item Finally we add up all of the terms that we wrote down.
\end{itemize}

\subsubsection{Examples}
Let's consider these examples:
$$ \sum_{i = 1}^{50} \pi \cdot i^2  = \pi0^2 + \pi1^2 + \pi2^2 + \ldots + \pi(50)^2$$
$$ \sum_{i = 0}^{3} (i^2 + 2i + 4) = (0 + 0 + 4) + (1+2+4) + (4+4+4) + (9+6+4) = 42$$
$$ \sum_{i = 0}^{3} (3i + 2)^2 = 4+25+63+121 = 214$$
