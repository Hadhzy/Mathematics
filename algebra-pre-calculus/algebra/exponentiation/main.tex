\section{Exponentiation}
When starting out with Exponentiation it is important first to understand the different parts of an exponential expression, so let's consider:
$$b^x = \underbrace{b \cdot b \cdot ... \cdot b}_{x \ times}$$
\begin{itemize}
  \item $b$ is the \textbf{base}.
  \item $x$ is the \textbf{exponent}.
\end{itemize}

As a reminder $ x^0 = 1 $

Let's discuss the different rules of Exponentiation:
\subsection{Product Rule}
To find the product of two exponential expression with the same base, add the exponents. 
$$ x^{n} \cdot x^{m} = x^{n+m} $$

\subsection{Quotient Rule}
When two exponential expressions with the same base are divided,  to find their quotient subtract their exponents. 
$$ \frac{x^n}{x^m} = x^{n-m} $$

\subsection{Power Rule}
When you raise an power to a power in an exponential expression to get the product, multiply the exponents
$$ (x^{n})^{m} = x^{n \cdot m} $$ 
$$ OR $$
$$ (x^{n})^{m} = \underbrace{x^n \cdot ... \cdot x^n}_{m \ times} $$
Here it is proven that the power rule is simply just the product rule.

\subsection{Negative Exponents}
When dealing with negative exponents this equation will apply: 
$$ x^{-n} = \frac{1}{x^n} $$

Since, 
$$ \frac{1}{x} = \frac{x^0}{x^1} = x^{0-1} = x^{-1} $$
It is important to note that this is just only one way of proving this, there are a couple more. 

\subsection{Fractional Exponents}
$$ \large x^{\frac{1}{n}} = \sqrt[n]{x} $$
The proof of this can be found in the next section \hyperref[sec:radicals]{Radicals(click to redirect)}.

\subsection{Additional rules}
Distribute an exponent over a product: $(x \ y)^{n} = x^{n} \ y^{n}$ \\
Distribute an exponent over a quotient: $  (\frac{a}{b})^{n} = \frac{a^{n}}{b^{n}}$

\begin{align*}
  \includegraphics[width=1.2\textwidth]{algebra-pre-calculus/algebra/exponentiation/laws-of-exponents.png}
\end{align*}

\newpage