\section{Preface}
Anytime you see a computer do something that appears to be intelligent, like recognising a face, playing a game better than people can, driving cars these are all examples of artificial intelligence (AI). 
We will begin this chapter with search. Like we have an AI or some kind of program that can search to solutions to some kind of problem. Whether it is trying to figure out how to get from point A to point B or how to win a game.
After that we will take a look at knowledge. We want our AI to be able to represent knowledge and use that knowledge to make decisions. 
Then we will explore the topic of uncertainty. We will look at how we can represent uncertainty and how we can use that to make decisions. By extension we will also talk about probability, and how computers can deal with uncertainty to be more intelligent and rational.
After that we will turn out attention to optimization. We will look at how we can optimize a solution. Especially when there might be multiple ways of solving that problem, but we are looking for a better way, or potentially the best way possible to solve that problem.
Then we will look at learning and machine learning. Of how when we have access to data, our computers can be programmed to be quite intelligent by learning from data, and learning from experience.
We will also look at that how computers are able to draw inspiration from human intelligence, looking at the structure of a human brain, and how neural networks can be a computer analog of that.
Finally we will take a look at language and how computers can understand language, and how they can generate language. This what we call natural language processing.